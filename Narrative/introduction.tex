\section{Introduction}

Because of the vast volume of data being produced by today's scientific simulations and experiments, lossy data compressor allowing user-controlled loss of accuracy during the compression is a relevant solution for significantly reducing the data size. However, lossy compressor developers and users are missing a tool to explore the features of scientific data sets and understand the data alteration after compression in a systematic and reliable way. To address this gap, we will design and implement a generic framework called Z-checker, such that the users/developers can conveniently select the best-fit, adaptive compressors for different data sets. On the one hand, Z-checker combines a battery of data analysis components for data compression. On the other, Z-checker is implemented as an open-source community tool to which users and developers can contribute and add new analysis components based on their additional analysis demands. In this paper, we present a survey of existing lossy compressors. Then we describe the design framework of Z-checker, in which we integrated evaluation metrics proposed in prior work as well as other analysis tools. Specifically, for lossy compressor developers, Z-checker can be used to characterize critical properties such as entropy, distribution, power spectrum, principal component analysis, and autocorrelation) of any data set to improve compression strategies. For lossy compression users, Z-checker can detect the compression quality (compression ratio, bit rate), provide various global distortion analysis comparing the original data with the decompressed data (PSNR, normalized MSE, rate-distortion, rate-compression error, spectral, distribution, derivatives) and statistical analysis of the compression error (maximum, minimum, and average error, autocorrelation, distribution of errors). Z-checker is also a flexible framework, whose assessment library can be extended with more plugins coded in other programming languages or libraries, such as R and FFTW3. Z-checker provides two alternative execution modes: offline mode or online mode. The offline mode allows users to do the data compression assessment based on generated raw data files or existing decompressed data files. The online mode allows performing an in-situ analysis with running applications. Z-checker also features a visualization interface displaying all analysis results in addition to some basic views of the data sets such as time series.
